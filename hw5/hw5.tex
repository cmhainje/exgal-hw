\documentclass{article}

\usepackage{amsmath,amssymb,graphicx}
\usepackage[parfill]{parskip}
\usepackage{hyperref}

\AtBeginEnvironment{quote}{\par\small}

\renewcommand{\d}{\mathrm{d}}


\begin{document}

{
	\centering
	\section*{Extragalactic Astrophysics HW5}
	Connor Hainje
	\vspace{2em}
	\par
}

\textbf{Structure Formation, Analytic Exercise \#5}.

\begin{quote}
	Consider a spherical region with mean overdensity $\bar\delta > 0$,
	within an expanding universe with no cosmological constant. As long as
	there is no \textit{shell crossing} --- that is, material at one radius
	does not catch up to material at another radius --- the equations
	governing the radius of this sphere over time are
	\begin{equation}
		\label{eq:given}
		\frac{\d^2 R}{\d t^2}
		= - \frac{G M(<R)}{R^2}
		= - \frac{4 \pi G}{3} \bar\rho \left(1 + \bar\delta\right) R
	\end{equation}
\end{quote}

\section*{part (a)}

\begin{quote}
	In terms of $\Omega_m$ at the present time, what is the condition that
	the spherical region will collapse on itself?
\end{quote}

Let's begin with the first equality in Equation~\ref{eq:given}. We can massage
it slightly to obtain
\begin{equation}
	\frac{\d^2 R}{\d t^2} + \frac{G M}{R^2} = 0.
\end{equation}
Note that we drop the $(<R)$ for the enclosed mass because the mass in this
spherical region is conserved (due to the assumption of no shell crossing), and
so $M$ is a constant.

We integrate this equation over $R$. The first term integrates as
\begin{equation}
	\int \frac{\d^2 R}{\d t^2} \,\d R
	= 
	\int \frac{\d^2 R}{\d t^2} \, \frac{\d R}{\d t} \,\d t
	=
	\frac{1}{2} \left( \frac{\d R}{\d t} \right)^2.
\end{equation}
The second term gives
\begin{equation}
	\int \frac{G M}{R^2} \,\d R
	=
	-\frac{G M}{R},
\end{equation}
and the integral of $0$ is a constant, which we'll call $E$. Thus,
\begin{equation}
	\frac{1}{2} \left( \frac{\d R}{\d t} \right)^2
	-\frac{G M}{R}
	= E.
\end{equation}

Mukhanov \S 1.2.3 points out that this is the same form as the calculation
of the escape velocity for an object to escape the gravitational field of some
body. In particular, $R$ can be seen as describing the height of the object.
For that scenario, $E > 0$ means that the object is able to escape, where $E <
0$ describes the object gaining some height before turning around and falling
back down. In our case, $R$ increasing to some maximum before turning around
and crashing to zero is exactly the behavior of collapse, so our condition is
that $E < 0$.

For $E < 0$ to be true, we must have $\dot R^2 < 2 G M / R$. We can use the fact
that we know the mass $M$ in terms of the mean density of the universe $\bar
\rho$, the mean overdensity of the sphere $\bar \delta$ and the radius of the
sphere $R$ to re-arrange this into
\begin{equation}
	\begin{aligned}
	\dot R^2
	&< \frac{2 G}{R} \frac{4 \pi}{3} \bar\rho \left(1 + \bar\delta\right) R^3 \\
	&< \frac{8 \pi G}{3} \bar\rho \left(1 + \bar\delta\right) R^2.
	\end{aligned}
\end{equation}
We also know that the matter density parameter today is given by
\begin{equation}
	\Omega_m = \frac{8 \pi G}{3 H^2} \bar \rho,
\end{equation}
which we can use to replace $\bar \rho$, yielding
\begin{equation}
	\dot R^2 < H^2 \Omega_m \left(1 + \bar\delta\right) R^2.
\end{equation}
Rearranging gives
\begin{equation}
	\frac{(\dot R / R)^2}{H^2} < \Omega_m \left(1 + \bar\delta\right).
\end{equation}
\textit{Initially}, the spherical overdensity will be expanding due to the
expansion of the universe, so $\dot R / R = H$ at $t = 0$. Since $\bar\delta$
describes the initial overdensity, the entire inequality is technically being
considered at the initial time, so we can use this! Cancelling these terms, we
finally obtain
\begin{equation}
	\Omega_m \left(1 + \bar\delta\right) > 1,
\end{equation}
as was seen in the notes.


\section*{part (b)}

\begin{quote}
	Demonstrate that the solutions to the above equation can be expressed as
	\begin{equation}
		\begin{gathered}
			R / R_m = (1 - \cos\eta) / 2, \\
			t / t_m = (\eta - \sin\eta) / \pi, \\
		\end{gathered}
		\label{eq:ansatz}
	\end{equation}
	where at time $t_m$ the sphere reaches its maximum radius $R_m$, before
	collapsing.
\end{quote}

First, let's analyze the ansatz. $t/t_m$ is a monotonically increasing function
of $\eta$, which on average increases by 1 for an increase in $\eta$ of $\pi$.
As such, we can think of $\eta$ as a parameterization of time. Then, $R / R_m$
is a simple cosine varying from $0$ to $1$ with a period of $2 \pi$. It starts
at zero, increases to $1$ at $\eta = \pi$ ($t = t_m$), then decreases back to
zero at $\eta = 2\pi$ ($t = 2 t_m$). I don't think it really makes sense to
continue the solution beyond $\eta = 2\pi$, so we'll limit our analysis to this
first period only.

OK, now let's try out the ansatz. Given the behavior we described above, we can
compute $E$. Because we know that $R$ turns around when $R = R_m$, $\d R/\d\eta
= 0$ at that point.\footnote{%
Actually, regardless of the ansatz we choose to take, this must hold for any
solution which is initially expanding, then collapsing, and the maximum radius
is labeled $R_m$.} 
Since $E$ is conserved, we can just compute it at this moment, and we know that
it will hold for all $\eta$. We find
\begin{equation}
	E = - \frac{G M}{R_m},
\end{equation}
which is certainly negative.

Next, we can evaluate the expression which we defined to be equal to $E$. To
start, let's evaluate $\d R/\d t$.
\begin{equation}
	\frac{\d R}{\d t}
	= \frac{\d \eta}{\d t} \frac{\d R}{\d \eta}
	= \frac{R'}{t'},
\end{equation}
where we are now considering $R$ and $t$ to be functions of $\eta$ alone, and
the prime denotes a derivative (w.r.t.~$\eta$). Taking these derivatives of the
ansatz, we find
\begin{equation}
	\frac{\d R}{\d t}
	= \frac{(R_m/2) \sin\eta}{(t_m/\pi) (1 - \cos\eta)}
	= \frac{\pi R_m}{2 t_m} \frac{\sin \eta}{1 - \cos\eta}.
\end{equation}
Now we can evaluate
\begin{equation}
\begin{aligned}
	E
	&= \frac{1}{2} \left(
		\frac{\pi R_m}{2 t_m} \frac{\sin \eta}{1 - \cos\eta}
	\right)^2 - \frac{2GM}{R_m} \frac{1}{1 - \cos\eta} \\
	&= \frac{1}{(1 - \cos\eta)^2} \left[
		\frac{\pi^2 R_m^2}{8 t_m^2} \sin^2 \eta
		- \frac{2 G M}{R_m} (1 - \cos \eta)
	\right].
\end{aligned}
\end{equation}
We require this to reduce to $E = -G M / R_m$. Let's keep manipulating this
expression until we get something obvious.
\begin{equation}
\begin{aligned}
	E
	&= \frac{1}{(1 - \cos\eta)^2} \left[
		\frac{\pi^2 R_m^2}{8 t_m^2} \sin^2 \eta
		- \frac{2 G M}{R_m} (1 - \cos \eta)
	\right] \\
	&= \frac{G M}{R_m} \frac{1}{(1 - \cos\eta)^2} \left[
		\frac{\pi^2 R_m^3}{8 G M t_m^2} (1 - \cos^2 \eta)
		- 2 (1 - \cos \eta)
	\right] \\
\end{aligned}
\end{equation}
Now, if $(\pi^2 R_m^3)/(8 G M t_m^2)$ were equal to 1, we would find
\begin{equation}
\begin{aligned}
	E
	&= \frac{G M}{R_m} \frac{1}{(1 - \cos\eta)^2} \left[
		(1 - \cos^2 \eta) - 2 (1 - \cos \eta)
	\right] \\
	&= -\frac{G M}{R_m} \frac{1}{(1 - \cos\eta)^2} \left[
		1 - 2 \cos \eta + \cos^2 \eta
	\right] \\
	&= -\frac{G M}{R_m}.
\end{aligned}
\end{equation}
So, the given ansatz describes a one-parameter family of collapsing solutions with
\begin{equation}
	E = - GM/R_m, \quad t_m^2 = \frac{\pi^2 R_m^3}{8 G M}.
\end{equation}
I'm not sure how to (or if one can) show this is the only family of collapsing
solutions, but we've certainly verified that is one such family.


\section*{part (c)}

\begin{quote}
	Show that at time $t_m$, the density of the sphere relative to the mean
	density of the universe will be $\rho_m / \bar \rho(t_m) = 9 \pi^2 /
	16$. 
\end{quote}

Taking $R_m$ as given, we can compute the (mean) density of the sphere at its
maximum radius by just dividing the constant mass $M$ by the volume:
\begin{equation}
	\rho_m
	= \frac{M}{V_m}
	= \frac{M}{ (4\pi/3) R_m^3 }
	= \frac{3 M}{4 \pi} \frac{\pi^2}{8 G M t_m^2}
	= \frac{3 \pi}{32 G t_m^2}.
\end{equation}

Next we need to handle the evolution of the mean density of the universe.
During matter domination, which is when I assume this problem is taking place, we have 
$\bar\rho \propto a^{-3}$, and $a \propto t^{2/3}$. Further, we know
\begin{equation}
	H^2 = \frac{8 \pi G}{3} \bar\rho
\end{equation}
under the assumption of a flat universe ($\bar\rho = \rho_c$), which is (I
think) the same as assuming matter domination. As usual, $H = \dot a/a$, and we
can take our proportionality relation for $a$ above to find $\dot a = 2 a / 3
t$. Plugging this in, we find
\begin{equation}
	\frac{4}{9t^2} = \frac{8 \pi G}{3} \bar\rho
\end{equation}
which reduces to
\begin{equation}
	\bar\rho = \frac{1}{6 \pi G t^2}.
	\label{eq:rho_bar}
\end{equation}

Putting these together gives
\begin{equation}
	\frac{\rho_m}{\bar\rho(t_m)}
	= \frac{3 \pi}{32 G t_m^2} \frac{6 \pi G t_m^2}{1}
	= \frac{9 \pi^2}{16}
\end{equation}
as desired.


\section*{part (d)}

\begin{quote}
	The collapse of the sphere will proceed in reverse, and will therefore
	take $t_m$ to do so. However, upon full collapse shell-crossing will
	occur, because the collisionless dark matter will pass through the
	origin and oscillate around it. This process can be modeled to derive
	the detailed structure of the resulting halo mass profile, but the
	virial theorem ($U = -2K$) can tell us about its overall size. Show
	that the final characteristic radius of the resulting
	\textit{virialized} halo is $R_\text{vir} = R_m/2$.
\end{quote}

The total energy $E$ is conserved. It's easy to compute at $t_m$ when $\d R/\d
t = 0$ and so the kinetic energy is zero. The potential energy is proportional
to $-GM/R_m$; for our purposes we can just drop this constant of
proportionality. Thus the total energy is $E = - GM/R_m$.

After virialization, we'll have $E = K + U$ and $U = - 2K$, but $E = - GM/R_m$
still. Thus, $K = GM/R_m$ and $U = -2 GM/R_m$. We also know that the
gravitational potential energy of this final configuration will be $U = -
GM/R_\text{vir}$, where $R_\text{vir}$ is the final radius of the virialized
halo. Comparing these two values of $U$, we see
\begin{equation}
	-2 GM/R_m
	= - GM/R_\text{vir}
	\implies
	R_\text{vir} = R_m / 2.
\end{equation}


\section*{part (e)}

\begin{quote}
	Show that the mean overdensity within the resulting halo is
	$\delta_\text{vir} = 18 \pi^2$.
\end{quote}

We can directly compute the mean density of the resulting halo as
\begin{equation}
	\rho
	= \frac{M}{V}
	= \frac{3M}{4\pi R_\text{vir}^3}
	= \frac{6M}{\pi R_m^3}
	= \frac{3 \pi}{4 G t_m^2}.
\end{equation}
We can use equation~\ref{eq:rho_bar} again to find the mean density of the
universe after virialization:
\begin{equation}
	\bar\rho = \bar\rho(2 t_m) = \frac{1}{24 \pi G t_m^2}.
\end{equation}
The ratio of these two gives
\begin{equation}
	1 + \delta_\text{vir}
	= \frac{\rho}{\bar\rho}
	= \frac{3 \pi}{4 G t_m^2} (24 \pi G t_m^2)
	= 18 \pi^2,
\end{equation}
which isn't exactly what we were asked to show but looks really close!


\section*{part (f)}

\begin{quote}
	By linearizing the equations~\ref{eq:ansatz}, show that the linearly
	extrapolated overdensity at the time of collapse is
	$\delta_\text{lin}(2 t_m) \approx 1.686$.
\end{quote}

We begin by series expanding equations~\ref{eq:ansatz} to obtain
\begin{gather}
	R/R_m
	= \frac{1}{2}(1 - \cos\eta)
	= \frac{\eta^2}{4} - \frac{\eta^4}{48} + \cdots, \\
	t/t_m
	= \frac{1}{\pi} (\eta - \sin\eta)
	= \frac{1}{\pi} \left(
		\frac{\eta^3}{6} - \frac{\eta^5}{120} + \cdots
	\right).
\end{gather}
I plugged the second series into Mathematica's \texttt{InvertSeries} function
to obtain
\begin{equation}
	\eta^2 = (6\pi t/t_m)^{2/3} \left[
		1 + \frac{1}{30} (6\pi t/t_m)^{2/3} + \cdots
	\right].
\end{equation}
Plugging this into the $R/R_m$ series gives
\begin{equation}
\begin{aligned}
	R/R_m
	&= \frac{1}{4} (6\pi t/t_m)^{2/3} \left[
		1 + \frac{1}{30} (6\pi t/t_m)^{2/3}
	\right] - \frac{1}{48} (6\pi t/t_m)^{4/3} + \cdots \\
	&= \frac{1}{4} (6\pi t/t_m)^{2/3} \left[
		1 - \frac{1}{20} (6\pi t/t_m)^{2/3}
	\right] + \cdots. \\
\end{aligned}
\end{equation}
Now we want to repeat part (e) using the linearized $R/R_m$. The mean density is
\begin{equation}
	1 + \delta_\text{lin}(t)
	= \frac{\rho}{\bar\rho(t)}
	= \frac{3M}{4\pi R(t)^3} (6\pi G t^2)
	= \frac{9 G M t^2}{2 R(t)^3}.
\end{equation}
Using the series expansion, we estimate
\begin{equation}
\begin{aligned}
	R^{-3}
	&= R_m^{-3} \frac{64}{(6 \pi t/t_m)^2} \left[
		1 - \frac{1}{20} (6\pi t/t_m)^{2/3}
	\right]^{-3} \\
	&= \frac{2}{9 G M t^2} \left[
		1 + \frac{3}{20} (6\pi t/t_m)^{2/3}
	\right.
\end{aligned}
\end{equation}
which can be inserted readily into $1 + \delta_\text{lin}$ to yield
\begin{equation}
	\delta_\text{lin}(t) = \frac{3}{20} (6\pi t/t_m)^{2/3}.
\end{equation}
Evaluating this at $t = 2 t_m$, we find
\begin{equation}
	\delta_\text{lin}(2 t_m) = \frac{3}{20} (12\pi)^{2/3} \approx 1.686,
\end{equation}
as desired.


\end{document}
