\documentclass{article}

\usepackage{amsmath,amssymb,graphicx}
\usepackage[parfill]{parskip}
\usepackage{hyperref}

\AtBeginEnvironment{quote}{\par\small}

\renewcommand{\d}{\mathrm{d}}


\begin{document}

{
	\centering
	\section*{Extragalactic Astrophysics HW7}
	Connor Hainje
	\vspace{2em}
	\par
}

\textbf{Dynamics, Analytic Exercise \#5}.

\begin{quote}
	Using the spherical symmetric first-order Jeans Equation,
	\begin{equation}
		\partial_r (n \sigma_{rr}^2)
		+ \frac{n}{r} \left[
			2 \sigma_{rr}^2
			- (\sigma_{\theta\theta}^2 + \sigma_{\phi\phi}^2)
		\right]
		= - n \partial_r \Phi,
	\end{equation}
	show that
	\begin{equation}
		M(<r) = - \frac{r \sigma_{rr}^2}{G} \left[
			\frac{\d \ln n}{\d \ln r}
			+ \frac{\d \ln \sigma_{rr}^2}{\d \ln r}
			+ 2 \beta
		\right].
	\end{equation}
\end{quote}

Note that, due to spherical symmetry, $n$ and $\sigma_{rr}^2$ and $\Phi$ are functions
only of $r$, and thus we can switch between partial and total derivatives
w.r.t.~$r$ without an issue.

Let's begin from the Jeans Equation. Under spherical symmetry, we have
$\sigma_{\theta\theta}^2 = \sigma_{\phi\phi}^2$, so
\begin{equation}
	\partial_r (n \sigma_{rr}^2)
	+ \frac{2 n}{r} \left(
		\sigma_{rr}^2 - \sigma_{\theta\theta}^2
	\right)
	= - n \partial_r \Phi.
\end{equation}
We can also introduce the anisotropy parameter $\beta \equiv 1 -
\sigma_{\theta\theta}^2/\sigma_{rr}^2$. Plugging this in simplifies the Jeans
Equation to
\begin{equation}
	\partial_r (n \sigma_{rr}^2)
	+ \frac{2 n}{r} \sigma_{rr}^2 \beta
	= - n \partial_r \Phi.
\end{equation}

Now we want to derive an expression for the total enclosed mass, which requires
us consider the relation between the gravitational potential $\Phi$ and the
mass density $\rho$. This relation is
\begin{equation}
	\nabla^2 \Phi = 4 \pi G \rho.
\end{equation}
The enclosed mass can then be written in terms of $\Phi$ as
\begin{align}
	M(<r)
	&= \int_0^r \d r' \ 4 \pi r'^2 \rho \\
	&= \int_0^r \d r' \ \frac{r'^2}{G} \nabla'^2 \Phi \\
	&= \frac{1}{G} \int_0^r \d r' \ \partial_{r'} r'^2 \partial_{r'} \Phi \\
	&= \frac{r^2}{G} \partial_r \Phi.
\end{align}
Now $\partial_r \Phi$ is a term in the Jeans Equation, which we can solve for
and plug in. This gives
\begin{align}
	M(<r)
	&= -\frac{1}{G} \left[
		2 r \sigma_{rr}^2 \beta
		+ \frac{r^2}{n} \partial_r (n \sigma_{rr}^2)
	\right] \\
	&= -\frac{r \sigma_{rr}^2}{G} \left[
		2 \beta
		+ \frac{r}{n \sigma_{rr}^2} \partial_r (n \sigma_{rr}^2)
	\right].
\end{align}
Finally, we can identify the logarithmic derivative:
\begin{equation}
	\frac{\d \ln f}{\d \ln x}
	= \frac{\d x}{\d \ln x} \frac{\d \ln f}{\d x}
	= \frac{x}{f} \frac{\d f}{\d x}.
\end{equation}
Making this replacement yields
\begin{equation}
	M(<r)
	= -\frac{r \sigma_{rr}^2}{G} \left[
		2 \beta
		+ \frac{\d \ln (n \sigma_{rr}^2)}{\d \ln r}
	\right]
\end{equation}
or, expanding using log rules,
\begin{equation}
	M(<r)
	= -\frac{r \sigma_{rr}^2}{G} \left[
		2 \beta
		+ \frac{\d \ln n}{\d \ln r}
		+ \frac{\d \ln \sigma_{rr}^2}{\d \ln r}
	\right],
\end{equation}
as desired.



\end{document}
